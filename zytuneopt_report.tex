\documentclass[%
	11pt,
	a4paper,
	utf8,
	%twocolumn
		]{article}	

\usepackage{style_packages/podvoyskiy_article_extended}


\begin{document}
\title{Пояснительная записка\\{\large к модулю подбора метаконфигурации решателя SCIP \\для MILP-проблем с <<короткой итерацией>>}}

\author{\itshape Подвойский А.О}


\date{}
\maketitle

\thispagestyle{fancy}

%Здесь приводятся заметки по специальным вопросам теории оптимизации

%\shorttableofcontents{Краткое содержание}{1}

\tableofcontents

\section{Вводные замечания}

Подбор стратегии поиска решения и гиперпараметров решателя имеет смысл только для MILP-проблем с <<короткой итерацией>>, то есть для проблем, у которых полное время поиска решения не превосходит некоторого порога времени, считающегося приемлемым временем ожидания (например, 30 секунд).

Запуск решателя SCIP с различными комбинациями параметров выполняется параллельно. 

Результаты сводятся в таблицу со следующими полями:
\begin{enumerate}
	\item наименование проблемы (\verb|problem_name|, \verb|<string>|),
	
	\item наименование стратегии поиска решения (\verb|strategy_name|, \verb|<string>|),
	
	\item количество бинарных переменных до пресолвинга (\verb|n_bin_vars_before_presolving|, \verb|<int>|),
	
	\item количество целочисленных переменных до пресолвинга (\verb|n_int_vars_before_presolving|, \verb|<int>|),
	
	\item количество вещественных переменных до пресолвинга (\verb|n_cont_vars_before_presolving|, \verb|<int>|),
	
	\item количество ограничений до пресолвинга (\verb|n_conss_before_presolving|, \verb|<int>|),
	
	\item количество бинарных переменных после пресолвинга (\verb|n_bin_vars_after_presolving|, \verb|<int>|),
	
	\item количество целочисленных переменных после пресолвинга (\verb|n_int_vars_after_presolving|, \verb|<int>|),
	
	\item количество вещественных переменных после пресолвинга (\verb|n_cont_vars_after_presolving|, \verb|<int>|),
	
	\item количество ограничений после пресолвинга (\verb|n_conss_after_presolving|, \verb|<int>|),
	
	\item группа гиперпараметров снижения размерности задачи (\verb|group_presolving|, \verb|<string>|); допустимые значения \verb|default|, \verb|aggressive| и \verb|fast|,
	
	\item группа гиперпараметров решения проблемы в релаксированной постановке (\verb|group_relax_method|, \verb|<string>|); допустимые значения \verb|simplex|, \verb|primal_simplex|, \verb|dual_simplex|, \verb|barrier|,
	
	\item группа гиперпараметров ветвления и разрешения конфликтов по бинарным переменным (\verb|group_preferbinary|, \verb|<bool>|); допустмые значения: \verb|TRUE| или \verb|FALSE|,
	
	\item группа гиперпараметров адаптации проблемы (\verb|group_adaptation_problem|, \verb|<string>|); допустимые значения: \verb|easycip|, \verb|feasibility|, \verb|optimality|, \verb|hardlp|, \verb|numerics|, \verb|off|,
	
	\item группа гиперпараметров снижения точности решения (\verb|group_reduced_tol|, \verb|<bool>|); допустимые значения: \verb|TRUE| или \verb|FALSE|,
	
	\item полное время расчета (\verb|total_time|, \verb|<float>|),
	
	\item значение целевой функции (\verb|objective|, \verb|<float>|),
	
	\item значение зазора (\verb|gap|, \verb|<float>|).
\end{enumerate}

Таким образом, получается 288 комбинаций значений гиперпараметров для одной стратегии поиска решения (для машины с 4 физическими ядрами, одной стратегии поиска решения и средней продолжительности работы решателя на одной проблеме 1 минута, время работы модуля составляет 72 минуты).

На базе получившейся сводной таблицы можно собрать финальную метаконфигурацию решателя и применить ее к другим проблемам рассматриваемого семейства.

\section{Предложения по формированию пространства гиперпараметров решателя}

Решатель SCIP 8.0.3 в общей сложности поддерживает 2729 параметров. Рассмотреть все возможные комбинации параметров вычислительно не осуществимо. Кроме того не все гиперпараметры решателя следует подбирать <<изолировано>>. К примеру, следующий набор гиперпараметров с указанными значениями имеет смысл только для \emph{агрессивного пресолвинга}
\begin{lstlisting}[
style = bash,
numbers = none
]
presolving/restartfac = 0.0125
presolving/restartminred = 0.06
constraints/setppc/cliquelifting = TRUE
presolving/boundshift/maxrounds = -1
presolving/qpkktref/maxrounds = -1
presolving/stuffing/maxrounds = -1
presolving/tworowbnd/maxrounds = -1
presolving/dualinfer/maxrounds = -1
presolving/dualagg/maxrounds = -1
presolving/redvub/maxrounds = -1
propagating/probing/maxuseless = 1500
propagating/probing/maxtotaluseless = 75
\end{lstlisting}

Таким образом, предлагается выделить следующие группы гиперпараметров решателя, объединенные по целевому контексту:
\begin{enumerate}
	\item \emph{Группа гиперпараметров снижения размерности задачи} (\verb|presolving|):
	\begin{itemize}
		\item По умолчанию (\verb|DEFAULT|)
\begin{lstlisting}[
style = ironpython,
numbers = none
]
# PySCIPOpt
model.setPresolve(pyscipopt.SCIP_PARAMSETTING.DEFAULT)
\end{lstlisting}
		
		\item Агрессивное (\verb|AGGRESSIVE|)
\begin{lstlisting}[
style = ironpython,
numbers = none
]
# PySCIPOpt
model.setPresolve(pyscipopt.SCIP_PARAMSETTING.AGGRESSIVE)
\end{lstlisting}
\begin{lstlisting}[
style = bash,
numbers = none
]
# ЗНАЧЕНИЯ ПАРАМЕТРОВ НЕ ИЗМЕНЯЮТСЯ
# SCIP> set presolving emphasis aggr
presolving/restartfac = 0.0125
presolving/restartminred = 0.06
constraints/setppc/cliquelifting = TRUE
presolving/boundshift/maxrounds = -1
presolving/qpkktref/maxrounds = -1
presolving/stuffing/maxrounds = -1
presolving/tworowbnd/maxrounds = -1
presolving/dualinfer/maxrounds = -1
presolving/dualagg/maxrounds = -1
presolving/redvub/maxrounds = -1
propagating/probing/maxuseless = 1500
propagating/probing/maxtotaluseless = 75
\end{lstlisting}
		    \item Быстрое (\verb|FAST|)
\begin{lstlisting}[
style = ironpython,
numbers = none
]
# PySCIPOpt
model.setPresolve(pyscipopt.SCIP_PARAMSETTING.FAST)
\end{lstlisting}
\begin{lstlisting}[
style = bash,
numbers = none
]
# ЗНАЧЕНИЯ ПАРАМЕТРОВ НЕ ИЗМЕНЯЮТСЯ
# SCIP> set presolving emphasis fast
constraints/varbound/presolpairwise = FALSE
constraints/knapsack/presolpairwise = FALSE
constraints/setppc/presolpairwise = FALSE
constraints/and/presolpairwise = FALSE
constraints/xor/presolpairwise = FALSE
constraints/linear/presolpairwise = FALSE
constraints/logicor/presolpairwise = FALSE
constraints/cumulative/presolpairwise = FALSE
presolving/maxrestarts = 0
propagating/probing/maxprerounds = 0
constraints/components/maxprerounds = 0
presolving/domcol/maxrounds = 0
presolving/gateextraction/maxrounds = 0
presolving/sparsify/maxrounds = 0
presolving/dualsparsify/maxrounds = 0
constraints/logicor/implications = FALSE
\end{lstlisting}
	\end{itemize}

    \item \emph{Группа гиперпараметров решения проблемы в релаксированной постановке}
\begin{lstlisting}[
numbers = none
]
lp/initalgorithm = s'implex  # default
                 = p'rimal simplex
                 = d'ual simplex
                 = b'arrier
                 = barrier with c'rossover
randomization/lpseed = 0  # default
\end{lstlisting}

    \item \emph{Группа гиперпараметров ветвления и разрешения конфликтов по бинарным переменным}
\begin{lstlisting}[
style = bash,
numbers = none
]
# Эти параметры <<включаются>> и <<выключаются>> только в паре
conflict/preferbinary = TRUE | FALSE
branching/preferbinary = TRUE | FALSE
\end{lstlisting}

    \item \emph{Группа гиперпараметров адаптации проблемы}
    \begin{itemize}
    	\item \verb|easycip|
\begin{lstlisting}[
style = bash,
numbers = none
]
# ЗНАЧЕНИЯ ПАРАМЕТРОВ НЕ ИЗМЕНЯЮТСЯ
# PySCIPOpt: model.setEmphasis(pyscipopt.SCIP_PARAMEMPHASIS.EASYCIP)
# SCIP> set emph easycip
# predefined parameter settings for easy problems
heuristics/clique/freq = -1
heuristics/completesol/freq = -1
heuristics/crossover/freq = -1
heuristics/gins/freq = -1
heuristics/locks/freq = -1
heuristics/lpface/freq = -1
heuristics/alns/freq = -1
heuristics/multistart/freq = -1
heuristics/mpec/freq = -1
heuristics/ofins/freq = -1
heuristics/padm/freq = -1
heuristics/rens/freq = -1
heuristics/rins/freq = -1
heuristics/undercover/freq = -1
heuristics/vbounds/freq = -1
heuristics/distributiondiving/freq = -1
heuristics/feaspump/freq = -1
heuristics/fracdiving/freq = -1
heuristics/guideddiving/freq = -1
heuristics/linesearchdiving/freq = -1
heuristics/nlpdiving/freq = -1
heuristics/subnlp/freq = -1
heuristics/objpscostdiving/freq = -1
heuristics/pscostdiving/freq = -1
heuristics/rootsoldiving/freq = -1
heuristics/veclendiving/freq = -1
constraints/varbound/presolpairwise = FALSE
constraints/knapsack/presolpairwise = FALSE
constraints/setppc/presolpairwise = FALSE
constraints/and/presolpairwise = FALSE
constraints/xor/presolpairwise = FALSE
constraints/linear/presolpairwise = FALSE
constraints/logicor/presolpairwise = FALSE
constraints/cumulative/presolpairwise = FALSE
presolving/maxrestarts = 0
propagating/probing/maxprerounds = 0
constraints/components/maxprerounds = 0
presolving/domcol/maxrounds = 0
presolving/gateextraction/maxrounds = 0
presolving/sparsify/maxrounds = 0
presolving/dualsparsify/maxrounds = 0
constraints/logicor/implications = FALSE
separating/maxbounddist = 0
constraints/and/sepafreq = 0
separating/aggregation/maxroundsroot = 5
separating/aggregation/maxtriesroot = 100
separating/aggregation/maxaggrsroot = 3
separating/aggregation/maxsepacutsroot = 200
separating/zerohalf/maxsepacutsroot = 200
separating/zerohalf/maxroundsroot = 5
separating/gomory/maxroundsroot = 20
separating/mcf/freq = -1
\end{lstlisting}

    \item \verb|feasibility|
\begin{lstlisting}[
style = bash,
numbers = none
]
# ЗНАЧЕНИЯ ПАРАМЕТРОВ НЕ ИЗМЕНЯЮТСЯ
# PySCIPOpt: model.setEmphasis(pyscipopt.SCIP_PARAMEMPHASIS.FEASIBILITY)
# SCIP> set emph feas
# predefined parameter settings for feasibility problems
heuristics/actconsdiving/freq = 20
heuristics/adaptivediving/freq = 3
heuristics/adaptivediving/maxlpiterquot = 0.15
heuristics/bound/freq = 20
heuristics/clique/freq = 20
heuristics/coefdiving/freq = 20
heuristics/completesol/freq = 20
heuristics/conflictdiving/freq = 5
heuristics/conflictdiving/maxlpiterofs = 1500
heuristics/conflictdiving/maxlpiterquot = 0.225
heuristics/crossover/freq = 15
heuristics/dins/freq = 20
heuristics/distributiondiving/freq = 5
heuristics/distributiondiving/maxlpiterofs = 1500
heuristics/distributiondiving/maxlpiterquot = 0.075
heuristics/dps/freq = 20
heuristics/farkasdiving/freq = 5
heuristics/farkasdiving/maxlpiterofs = 1500
heuristics/farkasdiving/maxlpiterquot = 0.075
heuristics/feaspump/freq = 10
heuristics/feaspump/maxlpiterofs = 1500
heuristics/feaspump/maxlpiterquot = 0.015
heuristics/fixandinfer/freq = 20
heuristics/fracdiving/freq = 5
heuristics/fracdiving/maxlpiterofs = 1500
heuristics/fracdiving/maxlpiterquot = 0.075
heuristics/gins/freq = 10
heuristics/guideddiving/freq = 5
heuristics/guideddiving/maxlpiterofs = 1500
heuristics/guideddiving/maxlpiterquot = 0.075
heuristics/zeroobj/freq = 20
heuristics/intdiving/freq = 20
heuristics/intshifting/freq = 5
heuristics/linesearchdiving/freq = 5
heuristics/linesearchdiving/maxlpiterofs = 1500
heuristics/linesearchdiving/maxlpiterquot = 0.075
heuristics/localbranching/freq = 20
heuristics/locks/freq = 20
heuristics/lpface/freq = 8
heuristics/alns/freq = 10
heuristics/nlpdiving/freq = 5
heuristics/mutation/freq = 20
heuristics/multistart/freq = 20
heuristics/mpec/freq = 25
heuristics/objpscostdiving/freq = 10
heuristics/objpscostdiving/maxlpiterofs = 1500
heuristics/objpscostdiving/maxlpiterquot = 0.015
heuristics/octane/freq = 20
heuristics/ofins/freq = 20
heuristics/padm/freq = 20
heuristics/proximity/freq = 20
heuristics/pscostdiving/freq = 5
heuristics/pscostdiving/maxlpiterofs = 1500
heuristics/pscostdiving/maxlpiterquot = 0.075
heuristics/randrounding/freq = 10
heuristics/rens/freq = 20
heuristics/reoptsols/freq = 20
heuristics/repair/freq = 20
heuristics/rins/freq = 13
heuristics/rootsoldiving/freq = 10
heuristics/rootsoldiving/maxlpiterofs = 1500
heuristics/rootsoldiving/maxlpiterquot = 0.015
heuristics/shiftandpropagate/freq = 20
heuristics/shifting/freq = 5
heuristics/trivial/freq = 20
heuristics/trivialnegation/freq = 20
heuristics/trustregion/freq = 20
heuristics/twoopt/freq = 20
heuristics/undercover/freq = 20
heuristics/vbounds/freq = 20
heuristics/veclendiving/freq = 5
heuristics/veclendiving/maxlpiterofs = 1500
heuristics/veclendiving/maxlpiterquot = 0.075
heuristics/rens/nodesofs = 2000
heuristics/rens/minfixingrate = 0.3
heuristics/crossover/nwaitingnodes = 20
heuristics/crossover/dontwaitatroot = TRUE
heuristics/crossover/nodesquot = 0.15
heuristics/crossover/minfixingrate = 0.5
heuristics/alns/trustregion/active = TRUE
heuristics/alns/nodesquot = 0.2
heuristics/alns/nodesofs = 2000
separating/maxrounds = 1
separating/maxroundsroot = 5
separating/aggregation/freq = -1
separating/mcf/freq = -1
nodeselection/restartdfs/stdpriority = 536870911
\end{lstlisting}

    \item \verb|optimality|
\begin{lstlisting}[
style = bash,
numbers = none
]
# ЗНАЧЕНИЯ ПАРАМЕТРОВ НЕ ИЗМЕНЯЮТСЯ
# PySCIPOpt: model.setEmphasis(pyscipopt.SCIP_PARAMEMPHASIS.OPTIMALITY)
# SCIP> set emph opt
# predefined parameter settings for proving optimality fast
separating/closecuts/freq = 0
separating/rlt/freq = 20
separating/rlt/maxroundsroot = 15
separating/disjunctive/freq = 20
separating/disjunctive/maxroundsroot = 150
separating/gauge/freq = 0
separating/interminor/freq = 0
separating/convexproj/freq = 0
separating/gomory/maxroundsroot = 15
separating/gomory/maxsepacutsroot = 400
separating/aggregation/maxsepacutsroot = 1000
separating/clique/freq = 20
separating/zerohalf/maxroundsroot = 30
separating/zerohalf/maxsepacutsroot = 200
separating/mcf/freq = 20
separating/mcf/maxsepacutsroot = 400
separating/eccuts/freq = 0
separating/eccuts/maxroundsroot = 375
separating/eccuts/maxsepacutsroot = 100
separating/oddcycle/freq = 0
separating/oddcycle/maxroundsroot = 15
separating/oddcycle/maxsepacutsroot = 10000
constraints/benderslp/sepafreq = 0
constraints/integral/sepafreq = 0
constraints/SOS2/sepafreq = 10
constraints/varbound/sepafreq = 10
constraints/knapsack/sepafreq = 10
constraints/knapsack/maxsepacutsroot = 500
constraints/setppc/sepafreq = 10
constraints/or/sepafreq = 10
constraints/xor/sepafreq = 10
constraints/conjunction/sepafreq = 0
constraints/disjunction/sepafreq = 0
constraints/linear/sepafreq = 10
constraints/linear/maxsepacutsroot = 500
constraints/orbitope/sepafreq = 0
constraints/logicor/sepafreq = 10
constraints/bounddisjunction/sepafreq = 0
constraints/benders/sepafreq = 0
constraints/pseudoboolean/sepafreq = 0
constraints/superindicator/sepafreq = 0
constraints/countsols/sepafreq = 0
constraints/components/sepafreq = 0
cutselection/hybrid/minorthoroot = 0.1
separating/maxroundsrootsubrun = 5
separating/maxaddrounds = 5
separating/maxcutsroot = 5000
constraints/linear/separateall = TRUE
separating/aggregation/maxfailsroot = 200
separating/mcf/maxtestdelta = -1
separating/mcf/trynegscaling = TRUE
branching/fullstrong/maxdepth = 10
branching/fullstrong/priority = 536870911
branching/fullstrong/maxbounddist = 0
branching/relpscost/sbiterquot = 1
branching/relpscost/sbiterofs = 1000000
branching/relpscost/maxreliable = 10
branching/relpscost/usehyptestforreliability = TRUE
\end{lstlisting}

    \item \verb|hardlp|
\begin{lstlisting}[
style = bash,
numbers = none
]
# ЗНАЧЕНИЯ ПАРАМЕТРОВ НЕ ИЗМЕНЯЮТСЯ
# PySCIPOpt: model.setEmphasis(pyscipopt.SCIP_PARAMEMPHASIS.HARDLP)
# SCIP> set emph hardlp
# predefined parameter settings for problems with a hard LP
heuristics/clique/freq = -1
heuristics/completesol/freq = -1
heuristics/crossover/freq = -1
heuristics/gins/freq = -1
heuristics/locks/freq = -1
heuristics/lpface/freq = -1
heuristics/alns/freq = -1
heuristics/multistart/freq = -1
heuristics/mpec/freq = -1
heuristics/ofins/freq = -1
heuristics/padm/freq = -1
heuristics/rens/freq = -1
heuristics/rins/freq = -1
heuristics/undercover/freq = -1
heuristics/vbounds/freq = -1
heuristics/distributiondiving/freq = -1
heuristics/feaspump/freq = -1
heuristics/fracdiving/freq = -1
heuristics/guideddiving/freq = -1
heuristics/linesearchdiving/freq = -1
heuristics/nlpdiving/freq = -1
heuristics/subnlp/freq = -1
heuristics/objpscostdiving/freq = -1
heuristics/pscostdiving/freq = -1
heuristics/rootsoldiving/freq = -1
heuristics/veclendiving/freq = -1
constraints/varbound/presolpairwise = FALSE
constraints/knapsack/presolpairwise = FALSE
constraints/setppc/presolpairwise = FALSE
constraints/and/presolpairwise = FALSE
constraints/xor/presolpairwise = FALSE
constraints/linear/presolpairwise = FALSE
constraints/logicor/presolpairwise = FALSE
constraints/cumulative/presolpairwise = FALSE
presolving/maxrestarts = 0
propagating/probing/maxprerounds = 0
constraints/components/maxprerounds = 0
presolving/domcol/maxrounds = 0
presolving/gateextraction/maxrounds = 0
presolving/sparsify/maxrounds = 0
presolving/dualsparsify/maxrounds = 0
constraints/logicor/implications = FALSE
branching/relpscost/maxreliable = 1
branching/relpscost/inititer = 10
separating/maxrounds = 1
separating/maxroundsroot = 5
\end{lstlisting}

    \item \verb|numerics|
    \end{itemize}
    \item \emph{Группа гиперпараметров точности решения (для численно нестабильных проблем)}
\begin{lstlisting}[
style = bash,
numbers = none
]
numerics/feastol = 1e-05
numerics/dualfeastol = 1e-06
numerics/epsilon = 1e-07
numerics/sumepsilon = 1e-05
\end{lstlisting}
\end{enumerate}

\remark{
Как показывают вычислительные эксперименты на проблемах с <<короткой итерацией>> поиск подгруппы первичных эвристик низкой эффективности не дает ощутмого прироста, а в большинстве случаев значительно снижает производительность. По этой причине первичные эвристики не принимают участи в подборе метаконфигурации решателя
}








\newpage
\listoffigures\addcontentsline{toc}{section}{Список иллюстраций}

% \listoftables\addcontentsline{toc}{section}{Список таблиц}

% Источники в "Газовой промышленности" нумеруются по мере упоминания 
\begin{thebibliography}{99}\addcontentsline{toc}{section}{Список литературы}	
	\bibitem{geron:ml-2018}{\emph{Жерон, О.} Прикладное машинное обучение с помощью Scikit-Learn и TensorFlow: концепции, инструменты и техники для создания интеллектуальных систем. -- СПб.: ООО <<Альфа-книга>>, 2018. -- 688 с.}
	
	\bibitem{soenen:effect-hyper-param-tuning:2021}{\emph{Soenen J. etc.} The Effect of Hyperparameter Tuning on the Comparative Evaluation of Unsupervised Anomaly Detection Methods, 2021}
\end{thebibliography}

\end{document}
